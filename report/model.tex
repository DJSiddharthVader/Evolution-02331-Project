We present a version of a modifier model to model the relationship between \ac{hgt} and \ac{crc} activity in response to environmental threats.
For simplicity we consider only a single antibiotic and a single bacteriophage and that the resistance allele confers full resistance to the antibiotic.
\subsection{Alleles}
We consider resistance allele $R$ that represents whether a bacterium posses a antibiotic resistance gene.
We also consider two modifier allele $C,H$ that represent the expression of \ac{crc} and \ac{hgt} machinery and  respectively
% Allele Definitions
\begin{table}[H]
    \centering
    \begin{tabular}{@{}lll@{}}
        \toprule
        \multicolumn{2}{c}{Allele} & Description \\
        \cmidrule(l){1-2}
        Major & Minor & \\
        \midrule
        $R$ & $r$ & has/does not have resistance gene \\
        $H$ & $h$ & \ac{hgt} machinery is expressed/not expressed  \\
        $C$ & $c$ & \ac{crc} is expressed/not expressed \\
        \bottomrule
    \end{tabular}
    \caption{Allele definitions}
\end{table}
\FloatBarrier

\subsection{Environment}
We consider 3 different environments
\begin{itemize}
    \item $E_n$: Neutral, no threats
    \item $E_b$: Bacteriophage, increased risk of contact with phage
    \item $E_a$: Antibiotic, increased risk of contact with antibiotic
\end{itemize}
We consider various scenarios that involve different functions that govern the transition from one environment to the next.
We consider 3 models of changing environments
\paragraph{Singular Threat:}
A singular event of antibiotic dosage or phage outbreak $e=(s,l)$ is defined by a length $l \in [0,M]$ and a start time $s \in [0,T]$.
The maximum length of a event is $M$ generations and the model runs over $T$ generations.
\paragraph{Cyclical Threat:}
Regular events of antibiotic dosage or phage outbreak every $2l$ generations.
More formally, each event is of length $l$ and we pick a set start times for each event $s_1=l, s_2=s_1+2l,\dots, s_i=s_{i-1}+2l$, and continue until $s_i+2l > T$.
Note you could very easily adapt this to randomly sample start times and lengths by alternatively sampling a first event $e_1 = (s_1,l_1) \in ([0,T],[0,M])$, a second event $e_2 = (s_2,l_2) \in ([s_1+l_1,T],[0,M])$ and so on until $s_i+l_i > T$.
\paragraph{Alternating Threat:}
It is defined similarly to the Cyclical Threat model but with switching between antibiotic dosage and phage outbreaks at each event.

\subsection{Fitness}
Now we want to know the fitness of each genotype in each environment, which we will define with the parameters $s_p,s_m$.
Here $s_p$ is the fitness benefit of protecting against the environmental threat (antibiotic or phage) and $s_m$ is the metabolic cost of expressing either \ac{hgt} or \ac{crc} machinery.
\footnote{Note that while we call $s_m$ the metabolic cost there are other fitness costs expressing \ac{crc} or \ac{hgt} machinery such as taking in toxic gene products, \ac{crc} autoimmunity, potential genome instability etc. but we group them all under the $s_m$ penalty.}
Both $H$ and $C$ express different proteins so any -HC genotype bacterium incurs a $2s_m$.
The $H$ allele acts like a modifier allele, it does not directly impact fitness outside of the metabolic cost but instead increases the success of gaining $R$ via HGT.
To ensure that we do see the invasion of protective alleles we also define $s_p >> s_m$ to better reflect the biological reality.
Given the above we define the fitness of each genotype in each environments in the following table
% Fitness
\begin{table}[H]
    \centering
    $\begin{array}{llll}
    \toprule
    Genotype & \multicolumn{3}{c}{Environment} \\
    \cmidrule(r){2-4}
        & E_n & E_b & E_a \\
    \midrule
    RCH & 1-2s_m & 1+s_p-2s_m & 1+s_p-2s_m \\
    RCh & 1- s_m & 1+s_p- s_m & 1+s_p- s_m \\
    RcH & 1- s_m & 1    - s_m & 1+s_p- s_m \\
    Rch & 1      & 1          & 1+s_p      \\
    rCH & 1-2s_m & 1+s_p-2s_m & 1    -2s_m \\
    rCh & 1 -s_m & 1+s_p- s_m & 1    - s_m \\
    rcH & 1- s_m & 1    - s_m & 1    - s_m \\
    rch & 1      & 1          & 1          \\
    \bottomrule
    \end{array}$
    \caption{Relative fitness values for each genotype in each environment}
    \label{ft}
\end{table}
\FloatBarrier

\subsection{Model Behaviour}
We are modelling bacteria, so we have an infinite haploid population that undergoes \emph{asexual} reproduction (no random mating).
We have 8 genotypes so we must keep track of 8 frequencies $[x_1(t),\dots,x_8(t)]$.
For every generation we have 1) gene transfer 2) mutation and 3) selection, here gene transfer is like a random mating step.
Note that $x_g$ represents the frequency of the genotype $g$.

\subsubsection{Gene Transfer}
The benefit of the $H$ allele over $h$ is defined later as it relates to the transfer step.
The probabilities of a successful transfer event between between two genotypes $G_r=$r- - and $G_R=$R- - (i.e. $r \to R$ transfer) are $g_H$ and $g_h$ for the -H-, -h- genotypes respectively where $g_H > g_h$.
Note that the $H$ allele matters for the both $G_r,G_R$ bacteria, so given two -H- bacteria the success probability is $2g_H$, given that only the receiving bacteria is $-H-$ we have just $g_H$ and given neither we have $g_h$.
The population frequency after selection is defined as
$$x_g^t = x_g + \delta(g)\sum_{x_R} x_{\neg g}x_R T(x_{\neg g},x_R)$$
where $T(g,\neg g)$ is defined as
\[T(g, \neg g) = \begin{dcases}
                     g_h & h \in g,\neg g \\
                     g_H & h \in g, H \in \neg g  \\
                     g_H & H \in g, h \in \neg g  \\
                    2g_H & H \in g,\neg g
                 \end{dcases}\]
and $\delta(g)$ is defined as
$$\delta(g) = \begin{cases} 1 & R \in g \\ -1 & R \notin g\end{cases}$$
Note that if $g=RCH$ then $\neg g=rCH$, defined similarly for $cH,Ch,ch$ genotype and $x_R = \{RCH,RcH,RCh,Rch\}$.

\subsubsection{Mutation}
Our mutation step is the same as in a regular modifier model.
A bacterium can evolve the resistance gene at a rate $P(r\to R) = \mu_R$ and a rate $\sqrt{\mu_R}$ with the selective pressure (in $E_a$).
Similarly it can lose the resistance gene at a rate $P(R\to r) =\mu_r$ and a rate $\sqrt{\mu_r}$ without the selective pressure (not in $E_a$).
The genotype frequencies after mutation are
$$x_g^s = (1-\mu(g,e))x_g^t + \mu(g,e)x_{\neg g}^t$$
where $\mu(g,e)$ is defined as
\[\mu(g,e) = \begin{dcases}
           \mu_R & r \in g, e \neq E_a \\
    \sqrt{\mu_R} & r \in g, e = E_a \\
           \mu_r & R \in g, e \neq E_a \\
    \sqrt{\mu_r} & R \in g, e = E_a \\
          \end{dcases}\]

\subsubsection{Selection}
Our selection step is also the same as in a regular modifier model using our fitness values.
The genotype frequencies after selection are
$$x_g' = \frac{x_g^sf(g,e)}{\bar{w}}$$
where $f(g)$ picks the correct fitness coefficient from Table \ref{ft} and $\bar{w}$ is the average fitness $\bar{w} = \sum_g x_g^sf(g)$.


